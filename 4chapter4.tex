\chapter{Control Software}
\label{chap4}

In this chapter, we provide an overview of the software deployed to control the motion of the robot and collect sensor readings in support of experiments.

%\section{Usage}
%\section{Maintenance}

\section{System Overview}
To support experiments, a sensor processing and robot control framework was written in C++ leveraging the libbarrett API provided by Barrett Technology Inc (Section~\ref{sec:libbarrett}). 
Our framework was developed exclusively on the internal PC of the WAM robot \cite{townsend1993mechanical}, the details of which are presented in Table~\ref{tbl:wam_specs}.
%For more details on the hardware specifications, please see Appendix~\ref{apx_B}.

\begin{table}[h]
\centering
\begin{tabular}{|l|l|}
\hline
Motherboard        & Aaeon PFM-540I                    \\
Processor          & 500 MHz AMD LX-800 x86-compatible \\
Memory             & 256 MB 200-pin DDR-333 SODIMM     \\
Linux distribution & Ubuntu 9.10                       \\
Linux kernel       & 2.6.31.4                          \\
Realtime           & Xenomai 2.5                       \\
Ethernet           & 10/100 Base-T                     \\
CANbus             & Peak PCAN-PC/104, 2 ports         \\ \hline
\end{tabular}
\caption{Specifications of the WAM Internal PC/104 \cite{wam_specs} used for framework development and robot control in support of experiments.}
\label{tbl:wam_specs}
\end{table}

\section{Libbarrett API}
\label{sec:libbarrett}

The API used to communicate with the robot is called libbarrett: a C++ library from Barrett Technology Inc. \cite{libbarrett}.
Our framework was tested using version 1.1.0 of the API.
The libbarrett API provides abstract control of the WAM and BarrettHand and allows them to be controlled in tandem.
Sample programs that perform simple control and sensor monitoring routines are provided, upon which our controller in the current study is based.

Libbarrett provides three high-level constructs to interface with the robot: (1) the WAM object (2) the Hand object and (3) the ProductManager object.
The WAM and Hand objects provide high-level control of the WAM and attached BarrettHand respectively.
The ProductManager provides access to the WAM's optional components, such as attached tools (e.g.  BarrettHand) and Force/Torque sensor.
The control software communicates with the Hand and WAM through high-level function calls to their respective libbarrett objects using a variety of pre-defined data structures. 
These datastructures are presented in Table~\ref{tbl:data_types}.
%Data types for interfacing with the WAM and BarrettHand are presented in Table~\ref{tbl:data_types}.

\begin{table}[h]
\centering
\begin{tabular}{| c | c | c |}
    \hline
    Name & Type & Unit \\
    \hline 
    Cartesian Position & cp\_type & Meters\\
    Joint Position & jp\_type & Radians\\
    Joint Velocity & jv\_type & Radians/s\\
    Joint Torque & jt\_type & Radians/s \\
    \hline 
\end{tabular}
\caption{Libbarrett data-types when communicating with the robot. The jp\_type for the WAM is a seven-dimensional vector whereas for the BarrettHand, it is a four-dimensional vector -- one entry for each finger and one for the spread.}
\label{tbl:data_types}
\end{table}


\section{WAM Control}

\paragraph{Cartesian-space}
Control in Cartesian space is a convenient way to prototype motions and is sufficiently repeatable for tasks which require low accuracy.
Accuracy of the Cartesian positioning of the WAM is advertised at two millimeters.
In practise, however, this accuracy depends largely on the joint-angle configuration of the WAM at the beginning of the Cartesian-space move.
In our experiments, position error could easily accumulate to reach as high as one centimeter, depending on the joint-angle position of the robot at the beginning of the Cartesian-space move.
These high errors may be due to imperfect inverse-kinematics currently available through the libbarrett API.
%Hand-tuned motions are more easily accomplished in Cartesian space when a specific motion within is desired and is effective only if the level of accuracy required is relatively low (0.5~1 cm).
Cartesian trajectories were not sufficiently repeatable in our experiments since we required a precise and highly repeatable motion (accurate to within 1 mm) to perform our task across various environments.
Moreover, the workspace accessible by rotating the wrist is limited by the angular configuration of its attaching joint. 
This means that certain wrist orientations are not repeatable if the inverse-kinematics solution provides differing joint-space positioning around the wrist.
These drawbacks prevent us from commanding the robot in Cartesian-space in our experiments.
%which is a limitation we have encountered in our experiments.
%Since motion of the robot is desired within a specific area of the workspace, the simplest and most straightforward 

\paragraph{Joint-space}
For accurate and repeatable sets of motions, the robot should be controlled directly in joint-space.
%This is problematic when developing motions by hand since specific Cartesian-space trajectories are extremely difficult to match via specific hand-tuned joint-angle trajectories.
The only issue with joint-space control is that the robot performs its task in Cartesian-space. 
Joint-angle trajectories that accomplish specific Cartesian-space tasks are difficult if not impossible to define manually.
This necessitates kinesthetic teaching where the robot is trained to perform its task in Cartesian space by a human expert manually moving the robot tool to perform a task, while corresponding joint-space trajectories are recorded by the robot.

\paragraph{Kinesthetic Teach-and-Play}
The libbarrett API comes equipped with kinesthetic teaching functionality via the teach-and-play module.
Teach-and-play records position information at the rate of 500Hz while a user physically moves the robot through the desired motions.
The robot can record its trajectory in either Cartesian-space or joint-space.
Again, due to imperfect inverse-kinematics, if a highly repeatable motion is required it is advisable to record trajectories in joint-space.
It becomes possible to execute highly repeatable Cartesian-space trajectories if the relative displacement from a known starting joint-angle position is small (i.e. a workspace of approximately $10cm^3$ around a starting joint-angle position).

\section{BarrettHand Control}

In this section, we introduce the details on control of the BH8-280 BarrettHand through the libbarrett API, as used in our experiments.

\paragraph{Velocity Move}
The simplest form of control of each finger of the Hand is to specify a direction and rate of travel of each of the joints in the Hand.
The joints will halt gracefully if they reach their limits or become obstructed before they reach these limits.

\paragraph{Trapezoidal Position Move}
If the Hand must be configured precisely (for e.g a pre-grasp posture), the user can specify desired joint-angles of each finger and spread.
The spread of the Hand refers to the single degree of freedom rotation of the first and third fingers about the palm.
Upon sending position commands the hand, the proximal finger joint angles or the angle of spread move toward the goal position via a trapezoidal velocity profile.
As with velocity control, the fingers will halt gracefully if the movement of the hand becomes obstructed.

\paragraph{High Control-rate Position Move}
High control-rate position moves provide an advanced alternative to simple trapezoidal moves.
Joint-angles can be specified to reach a desired pose, however each of the joints travel at maximum velocity until they reach the desired pose or become obstructed.
Care must be taken when sending these commands, as obstructions do not result in graceful halts and instead could cause damage to objects or the Hand itself.
It is necessary to ensure a clear path for each of the fingers and spread of the Hand before sending these commands.

\section{Realtime Systems}
\label{sec:realtime_eg}
Control of the robot in realtime must be done through the libbarrett realtime systems API.
Realtime control allows for complex and closed-loop motions since the output of each realtime system depends upon its inputs at each step of the realtime control loop at the rate of 500Hz.
Program~\ref{pgm:rtsys} provides an example program which defines such a realtime system in the C++ programming language.

\begin{landscape}
\begin{Program}[width=0.5\pagewidth]
\center
\begin{verbatim}
class WamSystem : public System {
public:     Input<jp_type> input;       // Obtain current joint-angles as input
public:     Output<jp_type> output;     // Provide updated joint-angles as output
protected:  Value* outputValue;         // Value that output reads
protected:  jp_type jp_offsets;         // Modifications to realtime motion feed
public:
    WamSystem(const string& sysName):   // All systems must define a name
        System(sysName), input(this), output(this, &outputValue){ 
            init_vec(&jp_offsets, 0);   // Initialize all offsets to 0
        }
    ~WamSystem(){ mandatoryCleanUp(); } // Mandatory destructor 
protected:
    jp_type jp_out;                     // Declare local copy of output data
    virtual void operate() {
        const jp_type& jp_in = input.getValue();    // Pull data from the input
        jp_offsets[5] += 0.001;         // Increase 6th joint-angle of WAM slightly
        jp_out = jp_in + jp_offsets;    // Modify wam joints by relative offsets
        outputValue->setData(&jp_out);  // Push data to subsequent system
    }
};
\end{verbatim}
  \caption{Example libbarrett realtime system written in C++ that controls the 6th joint of the WAM to increase indefinitely. Namespace references removed for brevity.}
  \label{pgm:rtsys}
\end{Program}
\end{landscape}

%%%%%%%%%%%%%%%%%%%%%%%%%%%

