\chapter{Hardware Overview}

%\section{Theory of Operation} 

%We begin this section with an overview of the equipment (i.e. the WAM and BarrettHand) followed by a brief look at how this equipment is monitored and controlled throughout the experiments.

\section{Whole Arm Manipulator (WAM)} 

The WAM is a 7 degree-of-freedom (DOF) robotic arm and wrist system from Barrett.
There are three input types when controlling the WAM: (1) joint position, (2) joint torque, and (3) Cartesian position.
All inputs are automatically translated into joint torques and fed to puck motors in the arm [40].
The simplest control input is Cartesian (X, Y, Z) however it does not guarantee exact positioning within the robot’s Cartesian workspace (approximated as a 2-meter diameter sphere), nor does this control scheme support any kind of collision avoidance.
In contrast, joint-space input offers complete and accurate control over all DOFs of the arm.

The WAM is PID-controlled in realtime mode whereas all other function is non-realtime.
This poses some restrictions on controller execution.
For example, output to the console during realtime operation is forbidden and forces the control program to halt.
The WAM comes with built-in gravity compensation mode, which, when active, adds a high level of compliance to the arm.

See Figure 3.2 for the WAM’s joint angle and velocity/acceleration limits and Figure 3.3 for further specifications.

\section{Force/Torque Sensor} 

The Barrett 6-Axis Force/Torque Sensor expands the force sensing capability of the WAM and BarrettHand systems.
The Force/Torque sensor processes signals from all the strain gages and outputs three forces and three torques within the Cartesian workspace of the WAM.
Full specifications are presented in Figure 3.5.

\section{BarrettHand} 

The BarrettHand is a 4 DOF, under-actuated, threefingered robotic hand from Barrett.
The DOFs are the joint positions at the base of each finger, and the spread of the first and second fingers around the circumference of the wrist.
See Figure 3.7 for details. 
The hand’s underactuation is possible via the TorqueSwitch technology, which is presented in Figure 3.6.
The hand is controlled in four fashions: (1) joint-torques; (2) joint-velocitie; (3)
joint-positions; and (4) high-level commands.
Jointposition commands are executed with a trapezoidal motion profile meaning that accelerations change instantaneously at the point the command is given.
High-level commands include commanding any or all fingers to open or close.
As with the WAM, all commands are eventually translated to joint-torques by the associated puck motors in the hand [41].
Apart from standard joint-encoders, the BarrettHand has optional sensors, which include tactile and strain-gage sensors.

\subsection{Tactile Sensors} 

Tactile sensors onboard the BarrettHand localize pressure across the palm and fingers.
They are arranged as four sets of cell arrays; one set within the distal segment of each finger as well as one set in the palm.
There are 96 cells in total (24 in each set).
See Figure 3.11 for full tactile sensor specifications.

\subsection{Strain-gage Sensors} 

Strain in the hand is measured as torques experienced about the distal joints of each of the hand’s fingers over a range of +/- 1 N-m.
Strain-gages measure the differential tension in the tendon-like structure running through each finger to the second joint.See Figure 3.8 for a schematic drawing of the joint-torque sensor system of the hand.
